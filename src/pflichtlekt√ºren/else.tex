

\subsection{Fräulein Else - schwerpunkt Psychoanalyse}
\label{sec:frauleinelse}
\point{Author:} Arthur Schnitzler

\subsubsection{Zusammenfassung}
\ttflushleft{
Else, 19 Jahre alt, aus gutem Haus, steht kurz vor dem finanziellen Ruin, da ihr Vater spielsüchtig ist und hohe Schulden hat. Sie ist bei ihrer Tante Emma in einem Hotel in Südtirol zu Gast. Begleitet wird sie von ihrem Cousin Paul, dessen Freundin Sissi Mohr, und dem Wiener Kunsthändler Herrn von Dorsday.

Während eines Tennis-Spiels denkt Else eifersüchtig an Paul. Ein Brief ihrer Mutter informiert sie, dass ihr Vater verhaftet werden könnte, weil er Herrn von Dorsday 30.000 Gulden schuldet. Else soll das Geld von Dorsday organisieren, sonst ist die Familie verloren.

Else ringt mit sich, findet keinen Ausweg, zieht ein schwarzes Kleid an und sucht Dorsday auf. Er macht ihr ein unmoralisches Angebot: Sie soll sich ihm nackt zeigen, als Gegenleistung für das Geld. Else ist innerlich zerrissen, wütend auf Dorsday und ihren Vater, empfindet Ekel und hat Selbstzweifel. Sie denkt an Suizid.

Sie nimmt Schlafmittel, legt einen Zettel mit Bedingungen vor Dorsdays Tür, sucht ihn später auf, lässt sich aber durch psychische Belastung vor Publikum gehen. Paul und Sissi kümmern sich um sie. Else halluziniert und der Text endet offen, ohne Klarheit über ihr Überleben.
}

\subsubsection{Analyse}

\point{Psychoanalyse}
\begin{itemize}
    \item Innerer Monolog als zentrales Stilmittel: zeigt Elses zerissenes Ich, widersprüchliche Gedanken, schnelle Themenwechsel und inneren Konflikte
    \item Else schwankt zwischen Fremdbestimmung (Erfüllung der Forderung von Dorsday, Familie retten) und dem Wunsch nach Selbstbehauptung
    \item Verschiedene Andeutungen deuten auf die Spielsucht ihres Vaters hin, die die familiäre Krise auslöst
    \item Else ist jung, hübsch, aber unverheiratet - ihre gesellschaftliche Situation verstärkt den Druck und die Angst von sozialem Abstieg
    \item Die Rolle von Paul und Sissi zeigt Elses sozialse Beziehungen, aber auch ihre Einsamkeit und Eifersucht
    \item Dorsday repräsentiert Machtmissbrauch und die Erniedrigung, der Else ausgeliefert ist
    \item Elses Gedanken über frühere sexuelle Erfahrungen und ihre ablehnende Haltung gegenüber Dorsday zeigt innere Konflikte und sexuelle Verwirrung
    \item Die Suizidgedanken und der Versuch mit Schlafmitteln spiegeln ihre psychische Überforderung und den Wunsch nach Flucht
    \item Das offene Ende unterstreicht die Unsicherheit und Tragik ihrer Sitation
\end{itemize}

