
\subsection{Mario und der Zauberer}
\label{sec:mario}
\point{Author:} Thomas Mann

\subsubsection{Rückblick:}

\ttflushleft{
Die Erzählung spielt in dem fiktiven italienischen Badeort Torre di Venere, wo eine deutsche Familie ihren Sommerurlaub verbringt. Von Anfang an werden sie feindselig behandelt: Nur Italienern wird erlaubt, auf der Hotelveranda zu sitzen, und als die Tochter der Familie niesen muss, beschwert sich ein Vertreter des römischen Hochadels über „Bazillenschleudern“. Obwohl ein Arzt die Gesundheit des Kindes bestätigt, wird die Familie in ein separates Gebäude verlegt. Daraufhin zieht sie in die Pension Eleonora, die von der freundlichen Signora Angioleri geführt wird – ein schönerer Ort mit besserem Essen.

Der Strandbesuch gestaltet sich ebenfalls unangenehm: Als die achtjährige Tochter ihren Badeanzug im Meer auswäscht, wird die Familie mit einer Geldstrafe von 50 Lire belegt. Diese Erlebnisse zeigen den vorherrschenden Nationalismus, Faschismus und die Feindseligkeit gegenüber ausländischen Touristen – besonders Deutschen, Engländern und Franzosen.

Die Familie überlegt abzureisen, doch Plakate für eine Show des Zauberers Cipolla wecken das Interesse der Kinder. Schließlich bleibt die Familie und besucht die Vorstellung.
Die Show beginnt mit Verzögerung. Cipolla tritt entspannt auf, raucht Zigaretten und lässt das Publikum warten. Als sich ein Mann meldet, wird er Cipollas erstes Opfer. Der Zauberer zwingt ihn mithilfe seiner Reitpeitsche, die Zunge herauszustrecken – trotz anfänglicher Weigerung. Cipolla wird mit einem buckligen und unangenehmen Äußeren beschrieben, das im Kontrast zu seiner dominanten Ausstrahlung steht.

Im Verlauf der Show führt Cipolla verschiedene Tricks und Manipulationen vor: einen Rechentrick, durch den er das Publikum beeindruckt, und er lässt denselben Mann vor Bauchschmerzen krümmen. Auch ein Kartentrick folgt, bei dem er scheinbar die gezogenen Karten errät. Er erkennt Signora Angioleri im Publikum und offenbart, dass sie Eleonora Duso begegnet ist – ohne dass sie es ausgesprochen hat.

In der Pause äußern die Eltern Bedenken wegen Cipollas manipulativem und demütigendem Verhalten. Die Kinder überreden sie jedoch, zu bleiben.

Nach der Pause hypnotisiert Cipolla einen weiteren Zuschauer, lässt ihn auf zwei Stühlen liegend wie ein Brett erscheinen und setzt sich auf ihn. Er bringt Signora Angioleri dazu, sich innerlich von ihrem Ehemann abzuwenden, und zwingt mehrere Zuschauer zu groteskem Tanz auf der Bühne, wobei er die Tanzenden immer wieder mit der Reitpeitsche antreiben muss. 
Zum Höhepunkt der Show ruft Cipolla den jungen Kellner Mario auf die Bühne, der sich zunächst im Hintergrund gehalten hatte. Er manipuliert ihn, enthüllt persönliche Details – wie seinen Liebeskummer – und zieht ihn immer tiefer in seinen Bann. Cipolla demütigt Mario öffentlich, bringt ihn dazu, sich für Silvestra (seine Geliebte) zu halten und ihn zu küssen. Das Publikum ist still, manche beginnen zu lachen. Cipolla wendet sich wieder den tanzenden Zuschauern zu, muss sie antreiben. In diesem Moment erwacht Mario aus der Hypnose, springt von der Bühne, zieht einen versteckten Revolver und erschießt Cipolla mit zwei Schüssen.

Im anschließenden Tumult verlässt die deutsche Familie wortlos die Veranstaltung und kehrt nach Hause zurück.
}


\subsubsection{Analyse:}
\begin{itemize}
    \item Mach und Manipulation:
    \begin{itemize}
        \item Cipolla steht für autoritäre Macht und die Gefährlichkeit von Manipulation. Er kontrolliert das Publikum durch Hypnose, Zwang und Demütigung
        \item Er zeigt, wie Menschen ihre Autonomie verlieren können und zu willenlosen Marionetten werden
        \item Cipollars körperliches Erscheinungsbild (bucklig, hässlich) konstrastiert mit seiner dominanten, bedrohlichen Ausstrahlung
    \end{itemize}
    \item Widerstand und Befreiung:
    \begin{itemize}
        \item Mario symbolisiert den Widerstand gegen Unterdrückung und Manipulation. Obwohl er zunächst unter Chipollas KOntrolle steht, gelingt ihm die Befreiung druch bewussten Widerstand (Schuss auf Cipolla)
        \item Marios Tat steht für den Bruch mit autoritärer Gewalt und das Wiedererlangen eigener Freiheit
    \end{itemize}
    \item Gesellschaftliche Atmosphäre:
    \begin{itemize}
        \item Der Handlungsort (italienischer Badeort unter faschistischem Einfluss) spiegelt Nationalismus, Ausgrenzung und Intoleranz wider (z.B. feindseliger Umgang mit ausländischen Touristen)
        \item Die Geschichte kritisiert eine Gesellschaft, die autoritäre Führer und Machtmissbrauch zulässt
    \end{itemize}
    \item Rolle des Publikums und der Zuschauer:
    \begin{itemize}
        \item Das Publikum wird zum Opfer von Chipollas Macht, passiv und teils amüsiert, aber auch hilflos
        \item Die Eltern zeigen Sekpsis, bleiben aber zögerlich und lassen die Kinder über den Verbleib entscheiden - Symbol für die Ohnmacht gegenüber autoritären Entwicklungen
    \end{itemize}
    \item Thematik Freiheit vs. Unterwerfung:
    \begin{itemize}
        \item Die Erzählung thematisiert den Konflikt zwischen individueller Freiheit und der Unterwerfung unter diktatorische Macht
        \item Cipolla verkörpert Unterdrückung, Mario den mutigen Widerstand
    \end{itemize}
\end{itemize}
