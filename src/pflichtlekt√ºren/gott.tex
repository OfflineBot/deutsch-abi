
\subsection{Der gute Gott von Manhattan}
\point{Author:} Ingeborg Bachmann


\subsubsection{Rückblick}
\ttflushleft{
 Das Hörspiel Der gute Gott von Manhattan von Ingeborg Bachmann erzählt die Geschichte einer radikalen, leidenschaftlichen Liebe zwischen Jan und Jennifer, die sich zufällig im New Yorker Grand Central Bahnhof begegnen. Ihre intensive Beziehung entzieht sich jeder gesellschaftlichen Norm – sie leben nur noch füreinander, außerhalb von Zeit, Verpflichtung und Funktion.

Diese absolute Form der Liebe wird vom „guten Gott von Manhattan“ als Bedrohung der gesellschaftlichen Ordnung gesehen. Er steht für Rationalität, Kontrolle und das Funktionieren der modernen Welt. In einem symbolisch aufgeladenen Akt lässt er Jennifer durch eine Bombe töten, um diese „gefährliche“ Liebe zu beenden.

Die Handlung entfaltet sich aus Rückblenden in einer fiktiven Gerichtsverhandlung, in der der gute Gott für seine Tat zur Rechenschaft gezogen wird. Dabei treten auch seine Helfer – zwei Eichhörnchen – als Zeugen auf, die die Beziehung von Jan und Jennifer beobachtet haben.

Im Laufe des Stücks wird deutlich, dass nicht nur ein individuelles Verbrechen vorliegt, sondern eine Systemkritik: Die moderne Gesellschaft lässt keine rauschhafte, bedingungslose Liebe zu. Selbst der Richter, der den Fall verhandelt, wird am Ende von der Argumentation des Täters überzeugt.
}

\subsubsection{Analyse}

\point{Liebe:}
\begin{itemize}
    \item Liebe macht willenlos ("Ich möchte jetzt alles so hinlegen..") und verhindert persönliche Bedürfnisse
    \item Jan verlässt Jennifer am Ende und überlebt $\rightarrow$ hat sich nicht vollens der Liebe hingegeben und lebt \textbf{sein} Leben weiter
    \item Liebe wird als "anderer Zustand" gesehen (keine Selbstverwirklichung möglich)
\end{itemize}

\point{Stockwerke:}
körperliche und geistige "Aufstieg" in höhere Stockwerke symbolisiert Isolation: je höher desto Isolierter

\point{Rolle des Guten Gottes, Richter und Einchhörnchen:}
\begin{itemize}
    \item "gute Gott" representiert kalte Vernunft, Ordnung und gesellschaftliche Kontrolle; sieht Liebe als abartig und zerstörerrisch, argumentiert mit gesellschaftlichem Allgemeinwohl
    \item Eichhörnchen (Billy \& Frankie) fungieren als Spione, Beobachter und Erfüllungsgehilfen - kleine Wesen die tödliche Konsequenz der Liebe vollziehen
    \item Der Richter: Zentrale Figur, bleibt passiv. Zeigt, dass zivilrechtliche Ordnung oft schwieriger anwendbar ist als moralische Prinzipien
\end{itemize}

