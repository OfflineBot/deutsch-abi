
\section{Pflichtlektüren}


\subsection{Mario und der Zauberer}
\label{sec:mario}
\point{Author:} Thomas Mann

\subsubsection{Zusammenfassung}
\ttflushleft{
Ein Urlauber in Italien gerät in eine politische und psychologische Krise, als er einer hypnotischen Show des Zauberers Cipolla beiwohnt. Cipolla nutzt seine Macht, um Menschen zu manipulieren und ihre Willensfreiheit zu zerstören. Das Stück thematisiert Machtmissbrauch, Manipulation und die Gefahr totalitärer Herrschaft.
}

\subsection{Der gute Gott von Manhattan}
\point{Author:} Ingeborg Bachmann

\subsubsection{Zusammenfassung}
\ttflushleft{
In diesem Hörspiel begegnen sich die beiden jungen Menschen Jan und Jennifer zufällig im Grand Central Bahnhof in New York. Ihre leidenschaftliche Liebe wird vom „guten Gott von Manhattan“ als unnatürlich betrachtet, was ihn dazu veranlasst, Jennifer mit einer Bombe zu töten. Die Handlung entfaltet sich in Rückblenden während einer Gerichtsverhandlung, in der der „gute Gott“ sich für seine Tat verantworten muss. Das Stück thematisiert die Spannung zwischen individueller Liebe und gesellschaftlichen Normen sowie die moralischen Implikationen von Macht und Kontrolle. 
}

\subsection{Fräulein Else - schwerpunkt Psychoanalyse}
\label{sec:frauleinelse}
\point{Author:} Arthur Schnitzler

\subsubsection{Zusammenfassung}
\ttflushleft{
Die Novelle schildert einen inneren Monolog einer jungen Frau, die in einer Notsituation moralisch erpresst wird, um die finanzielle Rettung ihrer Familie zu ermöglichen. Thema sind psychische Zerrissenheit, gesellschaftlicher Druck und die Sexualität der Frau im frühen 20. Jahrhundert.
}

