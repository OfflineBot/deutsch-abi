

\subsection{Fräulein Else - schwerpunkt Psychoanalyse}
\label{sec:frauleinelse}
\point{Author:} Arthur Schnitzler

\subsubsection{Zusammenfassung}
\ttflushleft{
Die Novelle schildert einen inneren Monolog einer jungen Frau, die in einer Notsituation moralisch erpresst wird, um die finanzielle Rettung ihrer Familie zu ermöglichen. Thema sind psychische Zerrissenheit, gesellschaftlicher Druck und die Sexualität der Frau im frühen 20. Jahrhundert.
}

\subsubsection{Analyse}

\point{Psychoanalyse}
\begin{itemize}
    \item Else steht zwischen Fremdbestimmung und Selbstbehauptung. Sie möchte helfen, ist aber überfordert mit der Situation
    \item Innerer Monolog zeigt ihre zerissenes Ich: widersprüchliche Gedanken, schelle Themenwechsel, innere Dialoge.
\end{itemize}

