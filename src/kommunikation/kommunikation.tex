
\section{Kommunikation und Kommunikationsmodelle}
$\Rightarrow$ Anweundung auf Texte üben (analoge/digitale Texte)


\subsection{Kommunikation}

\point{Definition:} Austausch von Informationen zwischen Sender und Empfänger. Kann bewusst oder unbewusst, verbal oder nonverbal erfolgen.

\point{Ziel:} Informationen sollen so rübergebracht werden wie sie gemeint sind.

\point{Kommunikationsarten:}
\begin{itemize}
    \item \textbf{Verbal:} Was? gesprochenes
    \item \textbf{Nonverbal:} Wie? Körpersprache, Mimik, Gestik, Bickkontakt, etc.
    \item \textbf{Paraverbal:} Wie? Tonfall, Lautstärke, Sprechtempo, etc.
\end{itemize}


\point{Begriffe:}

\begin{itemize}
    \item Metakommunikation: Reflektion über gespräch (macht es gerade überhaupt sinn?)
    \item Missverständnisse und Störung: Nimmt der andere es so auf wie ich es meine? Lässt sich durch Metakommunikation auflösen.
\end{itemize}



\subsection{Kommunikationsmodelle}
\label{sec:kommunikationsmodelle}

\subsubsection{Sender-Empfänger-Modell:}

\point{Funktionsweise:}
Kommunikation (wie Sprache) wird zuerst codiert und wird dann an den anderen Verschickt. Dieser decodiert und interpretiert dies.

\point{Kommunikationsmittel:}

\begin{itemize}
    \item \textbf{Verbal} das gesprochene Wort
    \item \textbf{Paraverbale:} die Artikulation
    \item \textbf{Nonverbale:} Gestik/Mimik; Körperhaltung; etc.
\end{itemize}

\point{TL;DR:} Nonverbale verstärkt Kommunikation. Alles gehört zu Kommunikation. Nicht nur Sprache.


\subsubsection{Vier-Ohren-Modell}

\point{Funktionsweise:}

Besteht aus:
\begin{itemize}
    \item Sachinhalt: Was ist der Inhalt?
    \item Appell: Was soll der Empfänger tun?
    \item Selbstkundgabe: Wie präsentiert sich der Sender? Was gibt er von sich preis?
    \item Beziehungshinweis: Was hat der Sender vom Empfänger? Welche Beziehungen haben beide zueinander?
\end{itemize}

\subsubsection{Axiome}

\point{1. Axiom:} Man kann nicht nicht kommunizieren

\point{2. Axiom:} Jede Kommunikation hat einen Inhalts- und einen Beziehungsaspekt. Dabei bestimmt der letzere den ersten

\point{3. Axiom:} Kommunikation beruht sich auf einem Wechselspiel aus Aktion und Reaktion.

\point{4. Axiom}: Es kann digital (ohne Interpretationsspielraum) und analog (mit Interpretationsspielraum) kommuniziert werden.

\point{5. Axiom}: Die Kommunikation kann symmetrisch (Gleichheit der Partner) oder komplementär (Unterschiedlichkeit der Partner) sein. 


\subsubsection{Eisberg}

\point{Funktionsweise:} Eisberg ist zu $80\%-90\%$ unter Wasser und $10\%-20\%$ sichtbar über Wasser. Bei Kommunikation sieht man nur die $10\%-20\%$ (spitze des Eisbergs) und $80\%-90\%$ der Kommunikation findet im nicht sichtbaren Teil ab.

Dabei sind die $10\%-20\%$ das gesprochene und die $80\%-90\%$ Emotionen, Werte, Gefühle, etc. die dabei mitspielen.


