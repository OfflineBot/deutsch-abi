
\section{Literaturgattungen}


\subsection{Lyrik}

\point{Beispiel:} Gedichte


\subsubsection{Gedichtsinterpretation}

\point{Wichtig:} Stilmittel müssen Hypothese unterstützen + Fachsprache (z.B. Verse statt Zeilen)

\point{zu Beachten..}
\begin{itemize}
    \item ..Formale Analyse (Metrum, Kadenz, Versbrüche)
    \item ..Sprache (Wortwahl, Satzbau, Wiederholungen, Zeitformen, Satzzeichen)
    \item ..Schlüsselbegriffe 
    \item ..Stilmittel
    \item ..formale Merkmale
    \item ..Wortfelder
    \item ..grammatikalische Auffälligkeiten
    \item ..Gedichtstyp
    \item ..Reimschema
    \item ..Entwicklung? Antithese? Gegenüberstellungen?
\end{itemize}


\subsection{Epik}

\point{Beispiel:} Roman, Kurzgeschichten, Märchen; (fließtexte die Geschichten erzählen)

\point{Wichtig:} Erzähltechnik, Figuren und Sprache müssen die Deutungshypothese stützen

\point{zu Beachten..}
\begin{itemize}
    \item ..Erzählform: Ich-Erzähler oder Er-/Sie-Erzähler
    \item ..Erzählverhalten: auktorial (allwissend), personal (an Figur gebunden), neutral (objektiv)
    \item ..Erzählperspektive und mögliche Wechsel
    \item ..Zeitgestaltung: Rückblenden, Vorausdeutungen, Zeitraffung/-dehnung
    \item ..Aufbau der Handlung: Einleitung, Höhepunkt, Wendepunkt, Schluss
    \item ..Erzähltempo und -reihenfolge
    \item ..Figurenkonstellationen und Konflikte
    \item ..sprachliche Mittel: Symbolik, Leitmotive, Wortwahl
    \item ..Ort, Zeit und Atmosphäre
    \item ..zentrale Themen und Motive
\end{itemize}


\subsection{Dramatik}

\point{Beispiel:} Theaterstück, Tragödien, Komödien

\point{Wichtig:} Szenenaufbau, Dialoge und Sprache müssen zentrale Konflikte und Intentionen deutlich machen

\point{zu Beachten..}
\begin{itemize}
    \item ..Aufbau: klassisches Drama (5 Akte) oder offenes Drama
    \item ..dramatischer Konflikt (innerlich oder äußerlich)
    \item ..Szenenstruktur und Spannungsbogen
    \item ..Figurenanalyse: Eigenschaften, Beziehungen, Entwicklung
    \item ..Dialogformen: Monologe, Dialoge, Streitgespräche, Stichomythien
    \item ..Sprachliche Besonderheiten und rhetorische Mittel
    \item ..Bühnen- und Regieanweisungen
    \item ..gesellschaftlicher oder historischer Kontext
    \item ..Wirkung auf das Publikum
\end{itemize}
