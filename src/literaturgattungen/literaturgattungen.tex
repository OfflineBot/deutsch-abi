
\section{Literaturgattungen}

\subsection{Lyrik}

\point{Beispiel:} Gedichte

\subsubsection{Gedichtsinterpretation}

\point{zum Schreiben:} 
\begin{enumerate}
    \item Formale Analyse (Metrum, Kadenz, Versbrüche)
    \item Entwicklung? Antithese? Gegenüberstellungen?
\end{enumerate}

\point{Wichtig:} Stilmittel müssen Hypothese unterstützen + Fachsprache (z.B. Verse statt Zeilen)

\point{zu Beachten..}
\begin{itemize}
    \item ..Sprache (Wortwahl, Satzbau, Wiederholungen, Zeitformen, Satzzeichen)
    \item ..Schlüsselbegriffe 
    \item ..Stilmittel
    \item ..formale Merkmale
    \item ..Wortfelder
    \item ..grammatikalische Auffälligkeiten
    \item ..Gedichtstyp
    \item ..Reimschema
\end{itemize}

\subsection{Epik}

\point{Beispiel:} Roman, Kurzgeschichten, Märchen; (fließtexte die Geschichten erzählen)

\subsection{Dramatik}

\point{Beispiel:} Theaterstück, Tragödien, Komödien


