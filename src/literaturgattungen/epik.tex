
\subsection{Epik}

\point{Beispiel:} Roman, Kurzgeschichten, Märchen; (fließtexte die Geschichten erzählen)

\point{Wichtig:} Erzähltechnik, Figuren und Sprache müssen die Deutungshypothese stützen

\point{zu Beachten..}
\begin{itemize}
    \item ..Erzählform: Ich-Erzähler oder Er-/Sie-Erzähler
    \item ..Erzählverhalten: auktorial (allwissend), personal (an Figur gebunden), neutral (objektiv)
    \item ..Erzählperspektive und mögliche Wechsel
    \item ..Zeitgestaltung: Rückblenden, Vorausdeutungen, Zeitraffung/-dehnung
    \item ..Aufbau der Handlung: Einleitung, Höhepunkt, Wendepunkt, Schluss
    \item ..Erzähltempo und -reihenfolge
    \item ..Figurenkonstellationen und Konflikte
    \item ..sprachliche Mittel: Symbolik, Leitmotive, Wortwahl
    \item ..Ort, Zeit und Atmosphäre
    \item ..zentrale Themen und Motive
\end{itemize}


