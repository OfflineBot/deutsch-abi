

%
% Ziele; Themen;
%

\section{Epochen}

\subsection{Romantik}
\point{Motive:}
\begin{itemize}
    \item Nacht (Gedichte wie Mondnacht; Der Kuss im Traume; Der Spinnern Nachtlied)
    \item Sehnsucht
\end{itemize}
FEHLT NOCH MEHR


\subsection{Literatur um 1900}


\subsubsection{Naturalismus}

\point{Merkmale:} Soziale Missstände, Industrialisierung, Armut, Arbeitsbedingungen

\point{Ziele:} Exakte, ungeschönigte Darstellung der Realität

\point{Merkmale:}
\begin{itemize}
    \item $\texttt{Kunst} = \texttt{Natur} - x$ ($x$ sollte möglichst gering sein)
    \item Wissenschaftliche Genauigkeit (Verwissenschaftlichung der Kunst)
    \item Einfluss von Milieu und Vererbung (Betonung Einfluss des sozialen Umfelds und genetischer Veranlagung auf Individuum)
    \item Darstellung des Hässlichen
    \item Wahrheitsbegriff (Prinzipien der Naturwissenschaft werden auf Literatur übertragen)
    \item Sekundarstil (Erzählzeit = erzählte Zeit)
\end{itemize}

\point{Historischer Kontext:}
\begin{itemize}
    \item gesellschaftliche Umbrüche geprägt durch 
    \begin{itemize}
        \item Industrielle Revolution
        \item Verstädterung
        \item Landflucht
        \item soziale Probleme (Armut, miserable Arbeitsbedingungen)
    \end{itemize}
\end{itemize}


\subsubsection{Impressionismus}
\point{Merkmale:}
\begin{itemize}
    \item Fokus auf subjektive Warhnehmung und flüchtige Endrücke
    \item Darstellung von Stimmungen und Momentanaufnahme
    \item Häufige Themen: Natur, städtisches Leben, Licht- und Farbschattierungen
    \item Stilmittel:
    \begin{itemize}
        \item Methaper
        \item Synästhesie
        \item Onomatopoesie (Lautmalerei) und Literatur ist sehr bildhaft
    \end{itemize}
\end{itemize}


\subsubsection{Symbolismus}
\point{Themen:} Träume, Mythen, Emotionen, Unbewusstes

\point{Merkmale:}
\begin{itemize}
    \item Fokus auf das Unaussprechliche: tiefere Bedeutung hinter der Realität
    \item Stil: kusntvoll, mehrdeutig, symbolisch
    \item Einsatz von Symbolen (zentrales Merkmal) und Metaphern für tiefere Bedeutungen
    \item Ziel: Darstellung einer geheimnisvollen Kunstwelt
    \item Kunst sollte nur sich selbst verpflichtet sein ("L'art pour L'art")
\end{itemize}

