

%
% Ziele; Themen;
%

\section{Epochen}

\subsection{Romantik}

\point{Merkmale:}
\begin{itemize}
    \item Folge von Historischem Kontext $\rightarrow$ romantisches Ich schafft sich Fluchtträume in Idyllisch verklärte Natur
    \item Denken der Romantiker geprägt durch suche nach den Wurzeln der deutschen Geschichte und Kultur \\ 
        $\rightarrow$ wiederentdeckung des Mittelalters
        $\rightarrow$ interpretation des Mittelalters als Zeit der Einheit, Ordnung und kulturellen Blüte
    \item Fiktion eines Lebens in der geordneten, heilen Welt des Mittelalters
    \item romantische Sehnsucht verlangt nach unbekanntem, unendlichen ohne Erfüllung finden zu können
    \item romantisch schmachtende Liebe (unerreichbar)
    \item Betrachtung des subjetiven Inneren
    \item Entgrenzung (Grenzen werden aufgehoben. bsp: bei Nacht verschwimmt Himmel und Erde/Fantasie und Realität)
    \item Gesellschaft poetisch machen (progressive Universalpoesie)
    \item Auch einfachen Dingen einen Sinn geben $\rightarrow$ Romantisieren
    \item Wohlklang und schöne Worte sind von zentraler Bedeutung für romantische Literatur
\end{itemize}

\point{Motive:}
\begin{itemize}
    \item Nacht 
        \begin{itemize}
            \item Beispiele: Gedichte wie Mondnacht; Der Kuss im Traume; Der Spinnern Nachtlied
            \item Wirkung: umrisse Verschwimmen (wie bei Entgrenzung "Himmel und Erde")
            \item Fantasie
            \item Gehemnis
            \item Wunderbar
            \item Zeit der Träume
        \end{itemize}
    \item Sehnsucht
    \item Tod
    \item Liebe
    \item Gesang und Musik spielen wichtige Rolle
    \item Blaue Blume als Symbol der Romantik (da nicht oft vorkommend) $\rightarrow$ vorstellung Fantasie
\end{itemize}

\point{Historischer Kontext:}
\begin{itemize}
    \item nach befreiung von französischer Oberhoheit
    \item hoffnung auf Umgestaltung der politischen Verhältnisse \\
        $\rightarrow$ wurde nicht erfüllt
    \item durch Industrialisierung wurde Mensch zunehmen in seinem ökonomischen Nutzwert gesehen \\
        $\rightarrow$ Utopie der Selbstverwirklichung des Individuums verblasst
\end{itemize}

\point{Beispiele:} 
Reiselyrik: Fernweh, Sehnsucht nach aufbruch in von unbegrenztheit und freiheit bestimmtes Leben

\point{Extra:}
Epoche "Sturm und Drang" gehört auch zur Romantik


\subsection{Literatur um 1900}

\subsubsection{Naturalismus}

\point{Merkmale:} Soziale Missstände, Industrialisierung, Armut, Arbeitsbedingungen

\point{Ziele:} Exakte, ungeschönigte Darstellung der Realität

\point{Merkmale:}
\begin{itemize}
    \item $\texttt{Kunst} = \texttt{Natur} - x$ ($x$ sollte möglichst gering sein)
    \item Wissenschaftliche Genauigkeit (Verwissenschaftlichung der Kunst)
    \item Einfluss von Milieu und Vererbung (Betonung Einfluss des sozialen Umfelds und genetischer Veranlagung auf Individuum)
    \item Darstellung des Hässlichen
    \item Wahrheitsbegriff (Prinzipien der Naturwissenschaft werden auf Literatur übertragen)
    \item Sekundarstil (Erzählzeit = erzählte Zeit)
\end{itemize}

\point{Historischer Kontext:}
\begin{itemize}
    \item gesellschaftliche Umbrüche geprägt durch 
    \begin{itemize}
        \item Industrielle Revolution
        \item Verstädterung
        \item Landflucht
        \item soziale Probleme (Armut, miserable Arbeitsbedingungen)
    \end{itemize}
\end{itemize}


\subsubsection{Impressionismus}
\point{Merkmale:}
\begin{itemize}
    \item Fokus auf subjektive Warhnehmung und flüchtige Endrücke
    \item Darstellung von Stimmungen und Momentanaufnahme
    \item Häufige Themen: Natur, städtisches Leben, Licht- und Farbschattierungen
    \item Stilmittel:
    \begin{itemize}
        \item Methaper
        \item Synästhesie
        \item Onomatopoesie (Lautmalerei) und Literatur ist sehr bildhaft
    \end{itemize}
\end{itemize}


\subsubsection{Symbolismus}
\point{Themen:} Träume, Mythen, Emotionen, Unbewusstes

\point{Merkmale:}
\begin{itemize}
    \item Fokus auf das Unaussprechliche: tiefere Bedeutung hinter der Realität
    \item Stil: kusntvoll, mehrdeutig, symbolisch
    \item Einsatz von Symbolen (zentrales Merkmal) und Metaphern für tiefere Bedeutungen
    \item Ziel: Darstellung einer geheimnisvollen Kunstwelt
    \item Kunst sollte nur sich selbst verpflichtet sein ("L'art pour L'art")
\end{itemize}

