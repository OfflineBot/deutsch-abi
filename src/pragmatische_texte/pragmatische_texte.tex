
\section{Pragmatische Texte}

\point{Beispiele:} Kommentare, Reden, Leserbriefe, Essays

\subsection{Anaylse}

\point{Ablauf:}
\begin{enumerate}
    \item \textbf{Thema und Intention}: Überzeugend? Informierend? Appelierend? Kritisierend?
    \item \textbf{Adressat}: Zielgruppe ist wichtig für Wahl der Argumente
    \item \textbf{Argumentationsstruktur}: Linear, dialektisch oder kontrovers?
    \item \textbf{\hyperref[sec:argumentationstypen]{Argumentationstypen}}
    \item \textbf{Sprachliche Mittel}: Stilmittel + Sprachstil (sachlich, emotional, ...)
    \item \textbf{Einordnung und Bewertung}: Wie wirkungsvoll? Einseitig? 
\end{enumerate}


\subsubsection{Argumentationstypen}
\label{sec:argumentationstypen}
\begin{itemize}
    \item Faktenargument
    \item Normatives Argument
    \item Autoritätsargument
    \item Analogisierendes Argument
    \item Indirektes Argument
    \item Gefühlsbetontes (emotionales) Argument
\end{itemize}
