
\section{Pragmatische Texte}

\point{Beispiele:} Kommentare, Reden, Leserbriefe, Essays

\subsection{Anaylse}

\point{Ablauf:}
\begin{enumerate}
    \item \textbf{Thema und Intention:} Überzeugend? Informierend? Appelierend? Kritisierend?
    \item \textbf{Adressat:} Zielgruppe ist wichtig für Wahl der Argumente
    \item \textbf{Argumentationsstruktur:} Linear, dialektisch oder kontrovers?
    \item \textbf{\hyperref[sec:argumentationstypen]{Argumentationstypen}}
    \item \textbf{Sprachliche Mittel:} Stilmittel + Sprachstil (sachlich, emotional, ...)
    \item \textbf{Einordnung und Bewertung:} Wie wirkungsvoll? Einseitig? 
\end{enumerate}


\subsubsection{Argumentationstypen}
\label{sec:argumentationstypen}
\begin{itemize}
    \item \textbf{Faktenargument:} überprüfbare Tatsachen; besonders starke Wirkung weil glaubwürdig
    \item \textbf{Normatives Argument:} bezieht sich auf Normen und Werte (gesellschaftiliche Regeln)
    \item \textbf{Autoritätsargument:} bezieht sich auf Meinung eines Experten oder anerkannten Institution
    \item \textbf{Analogisierendes Argument:} vergleicht aktuelle Sitution mit einer anderen um diese verständlicher zu machen
    \item \textbf{Indirektes Argument:} entkräftet ein mögliches Gegenargument um die eigenen Position zu stärken
    \item \textbf{Gefühlsbetontes (emotionales) Argument:} Appelliert an Emotionen (Mitgefühl oder Wut)
\end{itemize}
