
\subsection{Stilmittel}
\label{sec:stilmittel}

\setlength{\tabcolsep}{4pt}

\definecolor{mygray}{HTML}{444444}
\arrayrulecolor{mygray}
\renewcommand{\arraystretch}{1.3}
\begin{tabularx}{\textwidth}{|c|X|}
    \hline
    Stilmittel & Definition \\
    \hline
    \hline
    Metapher & Bildhafte übertragung nicht näher bestimmter Eigenschaften eines Begriffes auf einen anderen \\
    \hline
    Personifikation & Bildhafte vermenschlichung von Tieren, Pflanzen, Objekten oder Ideen \\
    \hline
    Vergleich & Verknüpfung zweier Bedeutungsbereiche durch hervorhebung des Gemeinsammen \\
    \hline
    Euphemismus & Beschönigung (oftmals als Metapher) \\
    \hline
    Neologismus & Wortneuschöpfung \\
    \hline
    Alliteration & Gleicher Anfangsbuchstabe bei aufeinanderfolgenden Wörtern \\
    \hline
    Anapher & Wiederholung gleicher Wörter am Beginn eines Satzes oder Satzteils \\
    \hline
    Antithese & Gegenüberstellung von gegensätzlichen Begriffen oder Gedanken \\
    \hline 
    Ironie & Offensichtlich unwahre Behauptung, mit der das Gegenteil ausgedrückt wird. Das Gegenteil des Gemeinten wird zum schein behauptet \\
    \hline 
    Hyperbel & Starke übertreibung \\
    \hline
    Symbol & Sinnbild das über sich hinaus auf etwas Allgemeines verweißt. Oft eine Sache oder Farbe \\
    \hline
    Ellipse & Auslassung eines Satzteiles oder Wortes. Führt zu unvollständigem Satz \\
    \hline 
    Hypotaxe & Unterordnung von mehreren Nebensätzen unter Hauptsätze \\
    \hline
    Parallelismus & Wiederholung einer syntaktischen Struktur \\
    \hline
    Parataxe & Aneinanderreihung von kurzen Hauptsätzen \\
    \hline
    Rhetorische Frage & Frage bei der die Antwort schon bekannt ist \\
    \hline
    Enjambement & Zeilensprung (Satz wird über Versgrenze weitergeführt) \\
    \hline
\end{tabularx}
