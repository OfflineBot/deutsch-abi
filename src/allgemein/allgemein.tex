
\section{Allgemein}


\point{Gute Quellen:} \href{https://www.youtube.com/playlist?list=PLFQhWtJUIAq00wSOlfr-xGlOsD8IETL3G}{Stark Erklärt}

\subsection{Operatorenliste:}

\setlength{\tabcolsep}{4pt}

\definecolor{mygray}{HTML}{444444}
\arrayrulecolor{mygray}
\renewcommand{\arraystretch}{1.3}
\begin{tabular}{|p{2cm}|p{6.7cm}|p{2.5cm}|}
    \hline
    Name & Bedeutung & Anforderungsb. \\
    \hline
    analysieren & Text als Ganzes oder in Teilen nach bestimmten Aspekten erschließen & I, II, III \\
    \hline
    beurteilen & Einschätzung oder Urteil auf Basis von Kriterien & II, III \\
    \hline
    beschreiben & Nerkmale von Personen, Sachverhalten oder Situationen darstellen & I, II \\
    \hline
    darstellen & Verbindungen zwischen Inhalten oder Sachverhalten aufzeigen & I, II \\
    \hline 
    einordnen & Zusammenhänge zwischen Text und Kontext herstellen & I, II \\
    \hline
    erklären & Sachverhalte und Textaussagen differenziert darstellen & II \\
    \hline
    erörtern & Argumente zu einer These abwägen und beurteilen & I, II, III \\
    \hline 
    untersuchen & Texte kriterienorientiert bearbeiten & II \\
    \hline
\end{tabular}

\subsection{Ich Erzähler - Lyrisches Ich}

\point{Wann was?:}


\subsection{Fehlt noch:}
\begin{itemize}
    \item Filmanalyse (alle)
    \item Kurzprosa (überarbeiten)
    \item Lyrik
    \item Pflichtlektüren (überarbeiten)
    \item pragmatische Text
\end{itemize}


