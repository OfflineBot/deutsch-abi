
\subsection{Fachsprache}
\setlength{\tabcolsep}{4pt}

\definecolor{mygray}{HTML}{444444}
\arrayrulecolor{mygray}
\renewcommand{\arraystretch}{1.3}
\begin{tabularx}{\textwidth}{|c|p{2.5cm}|X|X|}
    \hline
    Thema & Begriff & Erklärung (Was benennt man?) & Funktion (Wofür steht es?) \\
    \hline
    \hline
    Epik & Ich-Erzähler & Erzähler spricht in der Ich-Form & Subjetive Sicht, begrenzter Blick \\
    \hline 
    & personaler Erzähler & Erzähler ist auf Figur beschränkt & Innenperspektive einer Figur \\
    \hline 
    & auktorialer Erzähler & allwissender Erzähler mit Kommentaren & Überblick, Leseführung \\
    \hline
    & Erzählverhältnis & Art, wie erzählt wird (auktorial, personal, neutral) & Analyse des Erzählstils \\
    \hline
    & Zeitstruktur & z.B. Rückblende, Vorausdeutung & Wirkung auf Spannung und Verständnis \\
    \hline
    & Erzählerrede / Figurenrede & Wer spricht? Erzähler oder Figur? & Einordnung der Perspektive \\
    \hline
    & Erlebte Rede / innerer Monolog & Gedanken der Figur in 3. oder 1. Person & Näh zur Figur, Emotionalität \\
    \hline
    Lyrik & lyrisches Ich & "Ich" im Gedicht & Sprecher, nicht mit Auto gleichzustellen \\
    \hline
    & Metrum & Versmaß, z.B. Jambus & Rhythmus und Klang \\
    \hline
    & Reimschema & z.B.  Paarreim (aabb), Kreuzreim (abab) & Struktur  und Wirkung des Gedichts \\
    \hline
    & Enjambement & Zeilensprung & erzeugt Spannung oder Zusammenhang \\
    \hline
    & Metapher & bildhafte Sprache & übertragene Bedeutung, Tiefe \\
    \hline
    & Personifikation & Vermenschlichung & lebendige Darstellung \\
    \hline
    & Kadenz & Versende (männlich/weiblich) & bestimmt Klangwirkung \\
    \hline
    & Alliteration & gleicher Anlaut & Betonung, Klang \\
    \hline
    Dramatik & Monolog/Dialog & Selbstgespräch / Wechselrede & Charakterdarstellung, Konflikt \\
    \hline
    & Akt / Szene & Gliederung des Stücks & dramaturgischer Aufbau \\
    \hline
    & Botenbericht / Mauerschau & indirekte Darstellung & Spannung durch Bericht \\
    \hline
    & Konflikt & Gegensatz im Drama & zentrales Handlungselement \\
    \hline
    & offene / geschlossene Form & viele / unterschiedlich viele Orte, Zeiten & dramatische Struktur \\
    \hline
    Sprache allgemein & rhetorische Mittel & Stilmittel wie Metapher, Vergleich & Analyse sprachlicher Wirkung \\
    \hline
    & Parataxe / Hypotaxe & eingache / verschachtelte Sätze & Wirkung: klar vs komplex \\
    \hline
    & Wortfeld & Bedeutungsverwandte Wörter & zeigt Themen oder Stimmung \\
    \hline
    & Stil & sachlich, poetisch, dramatisch & Wirkung auf Leser \\
    \hline
\end{tabularx}

\setlength{\tabcolsep}{4pt}

\definecolor{mygray}{HTML}{444444}
\arrayrulecolor{mygray}
\renewcommand{\arraystretch}{1.3}
\begin{tabularx}{\textwidth}{|c|p{2.5cm}|X|X|}
    \hline
    Sachtextanalyse / Rede & These & Behauptung des Autors & Grundgedanke \\
    \hline
    & Parataxe/Hypothese & einfache / verschachtelte Sätze & Wirkung: klar vs. komplex \\
    \hline
    & Wortfeld & Bedeuntsverwandte Wörter & zeigt Themen oder Stimmung \\
    \hline
\end{tabularx}
