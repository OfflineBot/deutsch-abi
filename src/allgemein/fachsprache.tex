
\subsection{Fachsprache}
\setlength{\tabcolsep}{4pt}

\definecolor{mygray}{HTML}{444444}
\arrayrulecolor{mygray}
\renewcommand{\arraystretch}{1.3}
\begin{tabularx}{\textwidth}{|c|p{2.5cm}|X|X|}
    \hline
    Thema & Begriff & Erklärung (Was benennt man?) & Funktion (Wofür steht es?) \\
    \hline
    \hline
    Epik & Ich-Erzähler & Erzähler spricht in der Ich-Form & Subjetive Sicht, begrenzter Blick \\
    \hline 
    & personaler Erzähler & Erzähler ist auf Figur beschränkt & Innenperspektive einer Figur \\
    \hline 
    & auktorialer Erzähler & allwissender Erzähler mit Kommentaren & Überblick, Leseführung \\
    \hline
    & Erzählverhältnis & Art, wie erzählt wird (auktorial, personal, neutral) & Analyse des Erzählstils \\
    \hline
    & Zeitstruktur & z.B. Rückblende, Vorausdeutung & Wirkung auf Spannung und Verständnis \\
    \hline
    & Erzählerrede / Figurenrede & Wer spricht? Erzähler oder Figur? & Einordnung der Perspektive \\
    \hline
    & Erlebte Rede / innerer Monolog & Gedanken der Figur in 3. oder 1. Person & Näh zur Figur, Emotionalität \\
    \hline
    Lyrik & lyrisches Ich & "Ich" im Gedicht & Sprecher, nicht mit Auto gleichzustellen \\
    \hline
    & Metrum & Versmaß, z.B. Jambus & Rhythmus und Klan \\
    \hline
    
    \hline
\end{tabularx}
