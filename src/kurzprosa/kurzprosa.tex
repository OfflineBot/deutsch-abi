
\section{Kurzprosa}
Auch epische Texte.

\subsection{Kurzgeschichten}
\point{Merkmale:}
\begin{itemize}
    \item erzählende Texte
    \item unmittelbarer Einstieg / offenes Ende
    \item chronologischer Ablauf
    \item wenige Figuren / keine Exposition
    \item Personaler Erzähler
    \item alltagsnahe Umgebung / alltägliche Sprache
    \begin{itemize}
        \item ohne Namen
        \item Alltagsprobleme / Konflikt
    \end{itemize}
\end{itemize}

\point{Wichtig (muss rein):}
\begin{itemize}
    \item Figurenkonstellation (Hintergrundinformationen/\hyperref[sec:kommunikationsmodelle]{Kommunikationsmodelle})
    \item Erzählperspektive
    \item Erzählverhalten (Auktorialer Erzähler, usw.)
    \item Erzähltechnik (verhältnis von Erzählzeit zu erzählter Zeit)
    \item Stilmittel (+ Form und Stil)
    \item Ort und Zeit
\end{itemize}


\subsection{Parabeln}

\point{Definition:} Epischer/Fiktionaler Text. Leser überträgt Bildebene auf Sachebene.

\point{Merkmale:}
\begin{itemize}
    \item beschränkt sich auf das Wesentliche
    \item überwiegend kurze Texte
    \item Anhand eines erfundenen Einzelfalls wird eine verallgemeinernde Aussage dargestellt
    \item Leser muss sich "Sachebene" selbst erschließen (im Wortlaut Erzähltes $\neq$ was gemeint ist) \\
    $\rightarrow$ tertium comparationis
    \item traditionell:
    \begin{itemize}
        \item Ziel: Leser $\rightarrow$ Lehre / erziehen (klare didaktische Funktion)
        \item Beispiel: Ringparabel
    \end{itemize}
    \item modern:
    \begin{itemize}
        \item Autor und Leser auf Augenhöhe
        \item stellt Problem dar ohne Antwort zu kennen
    \end{itemize}
    \item Verweis auf \hyperref[sec:frauleinelse]{Fräulein Else}
\end{itemize}

\point{Mögliche Betrachtungsweisen:}
\begin{itemize}
    \item Biographisch
    \item Staatskritisch
    \item Religiös/Theologisch
    \item Gesellschaftilich (villt)
\end{itemize}

\point{Extra:} Bei Kafkas Texten bekommt man häufig einen guten Zugang indem man sich überlegt, was von Kafkas Biographie zum Text passt, bzw. gehört


\subsection{Novelle}

\point{Definition:} epischer Text (auch ähnlichkeiten zur Dramatik) der etwas länger als eine Kurzgeschichte ist aber deutlich Kürzer als ein Roman. Handelt von neuer unerhörter aber tatsächlich möglicher Einzelbegebenheit.

\point{Merkmale:}
\begin{itemize}
    \item kurzer epischer Text
    \item geradliniger Ablauf
    \item Wendepunkte
    \item oft einzelner Konflikt
    \item objektiver Berichtsstil, keine Einmischung des Erzählers, ohne epische Breite oder Charakterausmalung
    \item Rahmen- und Binnenerzählung
    \item geraffte Exposition
    \item konzentriert herausgebildete Peripatie
    \item Abklingen, welches die Zukunft der Person mehr ahnungsvoll andeuten als gestalten kann
    \item Verweis auf \hyperref[sec:mario]{Mario und der Zauberer}
\end{itemize}

