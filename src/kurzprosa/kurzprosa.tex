
\section{Kurzprosa}
Auch epische Texte.

\subsection{Kurzgeschichten}
\point{Merkmale:}
\begin{itemize}
    \item offener Anfang/Ende
    \item chronologischer Ablauf
    \item wenige Figuren
    \item alltagsnahe Umgebung
    \begin{itemize}
        \item ohne Namen
        \item Alltagsprobleme
    \end{itemize}
\end{itemize}


\subsection{Parabeln}
\point{Merkmale:}
\begin{itemize}
    \item beschränkt sich auf das Wesentliche
    \item Leser muss sich "Sachebene" selbst erschließen
    \item traditionell:
    \begin{itemize}
        \item Ziel: Leser $\rightarrow$ Lehre / erziehen
        \item Beispiel: Ringparabel
    \end{itemize}
    \item modern:
    \begin{itemize}
        \item Autor und Leser auf Augenhöhe
        \item stellt Problem dar ohne Antwort zu kennen
    \end{itemize}
    \item Verweis auf \hyperref[sec:frauleinelse]{Fräulein Else}
\end{itemize}


\subsection{Novelle}
\point{Merkmale:}
\begin{itemize}
    \item kurzer epischer Text
    \item geradliniger Ablauf
    \item Wendepunkte
    \item oft Konflikte
    \item Rahmen- und Binnenerzählung
    \item Verweis auf \hyperref[sec:mario]{Mario und der Zauberer}
\end{itemize}
