
\section{Filmanalyse}


\subsection{Kurzfilm}

\point{Merkmale:}
\begin{itemize}
    \item Erzählzeit $\neq$ Erzählte Zeit
    \item Handlungsorte (symbolische Bedeutung?)
    \item gestaltung der Handlungsorte
    \item 5 Akt Modell
    \begin{enumerate}
        \item Problementfaltung (Figur, Handlungsort, Zeit, Motiv)
        \item Steigerung der Handlung
        \item Krise und Umschwung
        \item Retardierung: Der Ausgang ist absehbar, wird aber durch verschiedene Umstände noch zurückgehalten
        \item Happy End oder Katastrophe
    \end{enumerate}
    \item Handlungsverlauf
    \item Geräusche
    \item Kamera (Kameraeinstellung/bewegung)
\end{itemize}

\subsection{filmische Mittel}
\begin{itemize}
    \item \textbf{Einstellung:} 
        \begin{itemize}
            \item Totale: z.B. Landschaft
            \item Halbtotale: z.B. ganze Person(en)
            \item Halbnah: z.B. obere hälfte einer Person
            \item Nah: z.B. Gesicht
        \end{itemize}
    \item \textbf{Perspektiven:}
        \begin{itemize}
            \item Vogelperspektive: Überblick
            \item Froschperspektive: von unten nach oben
            \item subjektive Kamera: POV
        \end{itemize}
    \item \textbf{Kamerabewegungen:} Schwenk, Zoom, Fahrt
    \item \textbf{Schnitt:} Tempowechsel, Rückblenden, Parallelmontagen; für Rhythmus, Spannung (auch Geschwindigkeit)
    \item \textbf{Licht:} Hell/Dunkel-Kontraste, Schatten; für Atmosphäre, spannung, etc.
    \item \textbf{Farbe:} Farbgebung kann Stimmung, Themen oder Figurencharakteristik unterstreichen
    \item \textbf{Ton und Musik:} schafft emotionale Tiefe, Geräusche dienen als Stilmittel oder Hinweise auf Bedeutung
\end{itemize}


\subsection{Hörspiel}

\begin{itemize}
    \item \textbf{Keine Bilder - alles passiert im Kopf:} gesamte Handlung muss über Geräusche, Sprache und Musik vermittelt werden. Zuhörer müssen sich Handlung selbst vorstellen.
    \item \textbf{Geräusche:} wichtig für Hörverstehen, z.B. Schritte, Türen, Wind. Geben Hinweis auf Ort, Zeit, Stimmung.
    \item \textbf{Sprecherstimmen:} Unterschiedliche Sprechlagen, Dialekte oder Sprechweisen helfen bei Figurenunterscheidung
    \item \textbf{Musik:} setzt Stimmungen, verbindet Szenen oder dient als Leitmotiv
    \item \textbf{Stille:} bewusst eingesetzt für Spannung oder ermöglicht nachzudenken
    \item \textbf{Narration:} kann Zeit, Ort, oder Gedanken vermitteln. Kann neutral oder Teil der Handlung sein
\end{itemize}

